\section{Introduction} % MH: I take responsibility for this...
%\section{Introduction \& Related Work}
\label{sec:intro}
We study the distance-$2$ coloring problem in the standard distributed \CONGEST model. Given a graph $G=(V,E)$, in the distance-$2$ coloring problem on $G$ (in the following just called \emph{d2-coloring}), the objective is to assign a color $x_v$ to each node $v\in V$ such that any two nodes $u$ and $v$ at distance at most $2$ in $G$ are assigned different colors $x_u\neq x_v$. Equivalently, d2-coloring asks for a coloring of the nodes of $G$ such that for every $u\in V$, all the nodes in the set $\set{u}\cup N(u)$ (where $N(u)$ denotes the set of neighbors of $u$) are assigned distinct colors. Further note that d2-coloring on $G$ is also equivalent to the usual vertex coloring problem on the graph $G^2$, where $V(G^2)=V$ and there is an edge $\set{u,v}\in E(G^2)$ whenever $d_G(u,v)\leq 2$. 
%\ym{I would have one sentence with the results here to excite the reader a bit} 


The \CONGEST model is a standard synchronous message passing model~\cite{peleg00}. The graph on which we want to compute a coloring is also assumed to form the network topology. Each node $u\in V$ has a unique $O(\log n)$-bit identifier $\ID(u)$, where $n=|V|$ is the number of nodes of $G$. Time is divided into synchronous rounds and in each round, every node $u\in V$ of $G$ can do some arbitrary internal computation, send a (potentially different) message to each of its neighbors $v\in N(u)$, and receive the messages sent by its neighbors in the current round. If the content of the messages is not restricted, the model is known as the \LOCAL model~\cite{linial92,peleg00}. In the \CONGEST model, it is further assumed that each message consists of at most $O(\log n)$ bits. 

% 
As our main result, we give an efficient $O(\log\Delta \log n)$-time randomized algorithm for d2-coloring $G$ with at most $\Delta^2+1$ colors, where $\Delta$ is the maximum degree of $G$. Further, we show that with slightly more colors, a similar result can also be achieved deterministically: We give a deterministic $\polylog n$-time algorithm to d2-color $G$ with $(1+\eps)\Delta^2$ colors for any $\eps>1/\polylog n$. 
Before discussing our results in more detail, we first discuss what is known for the corresponding coloring problems on $G$ and why it is challenging to transform \CONGEST algorithms to color $G$ into \CONGEST algorithms for d2-coloring. 

The distributed coloring problem is arguably the most intensively studied problem in the area of distributed graph algorithms and certainly also one of the most intensively studied problems in distributed computing more generally. The standard variant of the distributed coloring problem on $G$ asks for computing a vertex coloring with at most $\Delta+1$ colors. Note that such a coloring can be computed by a simple sequential greedy algorithm. In the following, we only discuss the work that is most relevant in the context of this paper, for a more detailed discussion of related work on distributed coloring, we refer to \cite{barenboimelkin_book,chang18_coloring,kuhn20_coloring}.

The $(\Delta+1)$-coloring problem was first studied in the parallel setting in the mid 1980s, where it was shown that the problem admits $O(\log n)$-time parallel solutions~\cite{alon86,luby86}. These algorithms immediately also lead to $O(\log n)$-round distributed algorithms, which even work in the \CONGEST model. In fact, even the following most simple algorithm $(\Delta+1)$-colors a graph $G$ in $O(\log n)$ rounds in the \CONGEST model: Initially all nodes are uncolored. The algorithm runs in synchronous phases, where in each phase, each still uncolored node $v$ chooses a uniform random color among its available colors (i.e., among the colors that have not already been picked by a neighbor) and $v$ keeps the color if no of its uncolored neighbors tries the same color at the same time~\cite{johansson99,BEPS12}.

Generally, the main focus in the literature on distributed coloring has been on the \LOCAL model, where by now the problem is understood relatively well. It was an important problem for a long time if there are similarly efficient deterministic algorithms for the distributed coloring problem (see, e.g., \cite{linial92,barenboimelkin_book,stoc17_complexity,derandomization_FOCS18}). This question was very recently resolved in a breakthrough paper by Rozho\v{n} and Ghaffari~\cite{RG19}, who showed that $(\Delta+1)$-coloring and many other important distributed graph problems have polylogarithmic-time deterministic algorithms in the \LOCAL model. The best randomized $(\Delta+1)$-coloring algorithm known %\mh{Isn't also optimal? Shouldn't we mention that?}\fk{It's not known to be optimal, but it probably is optimal in the following sense. It is known that the randomized complexity of $(\Delta+1)$-coloring on graphs of size $n$ is at least the deterministic complexity of $(\Delta+1)$-coloring on graphs of size $\sqrt{\log n}$, which is at least $\Omega(\log^* n)$. The paper gives an algorithm with complexity equal to the deterministic complexity of (degree+1)-list coloring on graphs of size $\poly\log n$. So, unless deterministic $(\Delta+1)$-coloring is easier than deterministic (degree+1)-list coloring, the algorithm is optimal. It however only achieves the optimal complexity by plugging in an optimal deterministic algorithm (which of course is not known).} 
in the \LOCAL model is by Chang, Li, and Pettie~\cite{chang18_coloring}, who show that the problem can be solved in time $\poly\log\log n$.\footnote{In \cite{chang18_coloring}, the complexity is given as $2^{O(\sqrt{\log\log n})}$. The improvement to $\poly\log\log n$ immediately follows from the recent paper by Rozho\v{n} and Ghaffari~\cite{RG19}. The same is true for the $n$-dependency in the \CONGEST model paper by Ghaffari~\cite{ghaffari19}, which is discussed below.} If the maximum degree $\Delta$ is small, the best known (deterministic) algorithm has a complexity of $O(\sqrt{\Delta\log\Delta}\cdot\log^*\Delta + \log^* n)$~\cite{fraigniaud16,BEG18}. We note that the $\log^* n$ term is known to be necessary due to a classic lower bound by Linial~\cite{linial92}. 
%\ym{How about changing the order of this paragraph and the one before? Right now it is congest-local-congest-local-congest}

\paragraph{From coloring to d2-coloring.} While most existing distributed coloring algorithms were primarily developed for the \LOCAL model, several of them directly also work in the \CONGEST model (e.g., the ones in \cite{alon86,luby86,linial92,johansson99,Kuhn2006On,barenboim10,BEK15,barenboim15,BEG18,kuhn20_coloring}). There is also some recent work, which explicitly studies distributed coloring in the \CONGEST model. In \cite{ghaffari19}, Ghaffari gives a randomized $(\Delta+1)$-coloring algorithm that runs in $O(\log\Delta) + \poly\log\log n$ rounds in the \CONGEST model. For the \CONGEST model, this is the first improvement over the simple randomized $O(\log n)$-round algorithms from the 1980s. Further, in another recent paper~\cite{det_congest_coloring}, by building on the recent breakthrough in the \LOCAL model~\cite{RG19}, it is shown that it is also possible to deterministically compute a $(\Delta+1)$-coloring in $\polylog n$ time in the \CONGEST model.

In the \LOCAL model, a single communication round on $G^2$ can be simulated in $2$ rounds on $G$ and therefore the distributed coloring problem on $G^2$ is at most as hard as the corresponding problem on $G$.\footnote{Note that not every graph $H$ is the square $G^2$ of some graph $G$ and thus, the coloring problem on $G^2$ might be easier than the coloring problem on $G$.} In the \CONGEST model, the situation changes drastically and it is no longer generally true that a \CONGEST algorithm on $G^2$ can be run at a small additional cost on the underlying graph $G$. In general, simulating a single \CONGEST round on $G^2$ requires $\Omega(\Delta)$ \CONGEST rounds on $G$. Note that even the very simple algorithm where each node picks a random available color cannot be efficiently used for d2-coloring as it is in general not possible to keep track of the set of colors chosen by some $2$-hop neighbor in time $o(\Delta)$. In some sense, our main technical contribution is an efficient randomized \CONGEST algorithm (on $G$) that implements this basic idea of iteratively trying a random color until all nodes are colored.

%We believe that distributed d2-coloring is an interesting and relevant problem for various reasons. 
\paragraph{Why d2-coloring?} Distributed d2-coloring is an interesting and important problem for several reasons. It is fundamental in wireless networking, where nodes with common neighbors \emph{interfere} with each other. Computing a frequency assignment such that nodes with the same frequency do not interfere with each other therefore corresponds to computing a d2-coloring of the communication graph \cite{KMR01}. Computing a coloring in a more powerful model (\congest) than it would be used in (wireless channels) is in line with current trends towards separation of control plane and data plane in networking. The d2-coloring problem also occurs naturally when single-round randomized algorithms are derandomized using the method of conditional expectation~\cite{derandomization_FOCS18}. Further, d2-coloring forms the essential part of \emph{strong coloring} hypergraphs, where nodes contained in the same hyperedge must be colored differently. One natural setting is when the nodes form a bipartite graph, with, say, ``task'' nodes on one side and ``resource'' nodes on the other side. We want to color the task nodes so that nodes using the same resource receive different colors. 


Finally, we can also view d2-coloring and other problems on $G^2$ as a way of studying \emph{communication capacity constraints} on nodes, where communication must go through intermediate relays. In fact, d2-coloring in \CONGEST is of special interest as it appears to lie at the edge of what is computable efficiently, i.e., in polylogarithmic time. Many closely related problems are either very easy or quite hard. The \emph{distance-$k$ maximal independent set} problem can easily be solved in $O(k\log n)$ time using Luby's algorithm~\cite{alon86,luby86}. The distance-3 coloring problem, however, appears to be hard. There is a simple reduction from the hardness of the $2$-party set disjointness problem~\cite{kalyanasundaram92,razborov92} to show that the closely related problem of verifying whether a given distance-$3$ coloring is valid requires $\Omega(\Delta)$ rounds, even on graphs where $\Delta=\Theta(n)$ (just think of a tree consisting of an edge $\set{a,b}$ and with $(n-2)/2$ leaf nodes attached to both $a$ and $b$). In fact, the classic set disjointness lower bound proof of Razborov~\cite{razborov92} implies that even verifying validity of a uniformly random coloring is hard.

\subsection{Contributions}

We provide different \CONGEST model algorithms to compute a d2-coloring of a given $n$-node graph $G=(V,E)$. If $\Delta$ is the maximum degree of $G$, the maximum degree of any node in $G^2$ is at most $\Delta + \Delta\cdot(\Delta-1)=\Delta^2$. As a natural analog to studying $(\Delta+1)$-coloring on $G$, we therefore study the problem of computing a d2-coloring with $\Delta^2+1$ colors. Although, there are extremely simple $O(\log n)$-time randomized algorithms for $(\Delta+1)$-coloring $G$, transforming similar ideas to d2-coloring turns out to be quite challenging. Our main technical contribution is an efficient randomized algorithm to d2-color $G$ with $\Delta^2+1$ colors. 

\begin{theorem}
\label{thm:d2ColoringRand}
There is a randomized \CONGEST algorithm that d2-colors a graph with $\Delta^2+1$ colors in $O(\log \Delta \log n)$ rounds, with high probability.
%\alg{Improved-d2-Color} properly d2-colors a graph with $\Delta^2+1$ colors in $O(\log \Delta \log n)$ rounds, w.h.p.
\end{theorem}

We outline the key ideas and challenges involved at the start of Sec.~\ref{sec:randAlg}. %The formal statement of this result is given by the following theorem, for an informal discussion of the key ideas and of the main challenges, we refer to respective overview in \fk{\Cref{sec:...}}

In addition to the randomized algorithm for computing a d2-coloring, we also provide two deterministic algorithms for the problem. The first one is obtained by a relatively simple adaptation of an $O(\Delta+\log^*n)$-time $(\Delta+1)$-coloring \CONGEST algorithm on $G$ to the d2-coloring setting~\cite{BEG18}.

\begin{theorem}
\label{thm:d2ColoringDelta}
There is a deterministic \CONGEST algorithm that d2-colors a graph with $\Delta^2+1$ colors in $O(\Delta^2+\logstar n)$ rounds.
%There is an algorithm that generates a $(\Delta^2 + 1)$-coloring of $G^2$ in time $O(\Delta^2 + \log^*n)$ in {\congest}.
\end{theorem}

Our second deterministic algorithm is more involved. From a high-level view, it uses ideas similar to several recent \CONGEST results ~\cite{det_congest_coloring,CPS17,DKM19,DISC18_DomSet}: 
%some other recent results on efficient deterministic algorithms for local graph problems in the \CONGEST model
With the algorithm of \cite{RG19}, one decomposes the graph into clusters of $\polylog n$ diameter that the problem can essentially be solved separately on each cluster (incurring a polylogarithmic overhead). On each cluster, one then uses the method of conditional expectation to efficiently derandomize a simple zero-round randomized algorithm. Unlike the algorithms in \cite{det_congest_coloring,CPS17,DKM19,DISC18_DomSet}, we do not use this general strategy to directly solve (a part of) the problem at hand (d2-coloring in our case). Instead, we apply the above strategy to implement a variant of the splitting problem discussed in \cite{BambergerGKMU19,stoc17_complexity}. By applying the splitting problem recursively, we partition the nodes $V$ into $\Delta/\polylog n$ parts such that a) we can use disjoint color palettes for the different parts, and b) we can efficiently simulate \CONGEST algorithms on $G^2$ on each of the parts (and these \CONGEST simulations can also efficiently be run in parallel on all the parts). By using slightly more colors, we can then also compute a d2-coloring in $\polylog n$ time deterministically.

\begin{theorem}[Simplified]
\label{thm:G2withSplitting}
For any fixed constant $\eps>0$, there is a deterministic \CONGEST algorithm that d2-colors a graph with $(1+\eps)\Delta^2$ colors in $\polylog n$ rounds.
\end{theorem}

%\ym{I dislike the numbering of the theorems. Maybe call them 1.1, 1.2 and 1.3 to make them stick out of the sheer amount of other lemmata that we have. Usually I don't like the 1.x numbering but here it seems to make sense} MH: Agreed. Made them also of parallel form

The remainder of the paper is structured as follows. In \Cref{sec:randAlg}, we present our randomized algorithm and prove \Cref{thm:d2ColoringRand}, our main technical result. In \Cref{S:det-diam}, we present our deterministic algorithms, proving \Cref{thm:d2ColoringDelta,thm:G2withSplitting}. Note that because of space restrictions, many of the proofs appear in an appendix. 
%\mh{This paragraph is a bit sad as is. Also, can we just state that when proofs are missing, they appear in the appendix?}